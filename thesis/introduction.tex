\chapterwithtoc{Introduction}

Many real-world domains generate data with a skewed distribution of its classes. While slight
skewness does not usually pose any problem, once it reaches a certain degree, standard machine
learning classifiers, such as decision trees, support vector machines and linear classifiers,
become unable to learn reliably from the data~\cite{learning-from-imb-data}. In such cases,
classifiers tend to get biased toward the majority class and neglect the underrepresented one,
resulting in wildly varying accuracies on individual classes. It is not unusual to see accuracy in
the majority class close to 100\%, while only 0-10\% in the minority
class~\cite{learning-from-imb-data}. Such disproportion in performance poses a significant problem
in many areas where the minority class is of greater importance. One such example is the problem of
medical diagnosis of rare diseases, such as cancer. There is a massive prevalence of negative, i.e.
non-cancerous patients over positive ones and misclassification of a cancerous patient as
non-cancerous will result in overwhelming consequences. There are, however, many more areas where
data exhibit significant imbalance, for instance, anomaly detection, fraud detection and network
intrusion detection~\cite{gosain2017}.

There has been a great deal of interest in solving the imbalanced classification problem from the
community of researchers in recent years. Many approaches have been proposed to tackle the problem.
Among those approaches are resampling methods, cost-sensitive learning methods, ensemble methods,
kernel-based learning methods and active learning methods~\cite{learning-from-imb-data,
gosain2017}. We shall focus our attention on one particular approach relying on modifying the
dataset called resampling, but we shall briefly discuss other methods. The resampling approach does
not require modification of the machine learning algorithm and is thus suitable for use with
existing and well-known ones. However, as we shall see later, there are numerous proposed
strategies, and it is not immediately clear which strategies are preferable and what are the
reasons for that.

This thesis aims to systematically and robustly compare these methods and provide the source code
to allow the reproduction of the results obtained and discussed in the second half of the thesis.
Although there are already publications comparing some of these methods, their authors usually
restrict themselves to one group of the resampling methods – oversampling methods~\cite{gosain2017,
amin2016}. Additionally, they use a small number of datasets to evaluate the
methods~\cite{gosain2017, amin2016, barandela2004}. As far as the author is aware, there has not
been a published comparison covering sixteen different preprocessing methods over seventeen
datasets.

The structure of the thesis is as follows. We shall review three standard machine learning
algorithms heavily used in the resampling methods and introduce standard and
imbalanced-classification-specific evaluation metrics in Chapter~\ref{chapter:prerequisites}. In
Chapter~\ref{chapter:imb-classif}, we shall discuss the imbalanced classification, briefly touch
upon various possible approaches to solving this problem and then focus on one group of methods –
the resampling methods. We shall describe many resampling algorithms employing both oversampling
and undersampling approaches. We include visual illustrations to demonstrate the results of
applying the methods to an artificial dataset. Chapter~\ref{chapter:experiments} will introduce the
framework we designed to effectively compare preprocessing methods over provided datasets, give
technical details about the conditions under which the experiments were run and finally describe
the critical components of the framework in greater detail. Chapter~\ref{chapter:results} will
present the results obtained. Lastly, we shall conclude the thesis and describe the future work.
