% Configuration of the title page
\def\ThesisTitleStyle{mff}

\def\UKName{Charles University in Prague}
\def\UKFaculty{Faculty of Mathematics and Physics}

\def\ThesisTypeTitle{BACHELOR THESIS}
\def\ThesisType{bachelor }

\def\ThesisTitle{Data Preprocessing Strategies in Imbalanced Data Classification}
\def\ThesisAuthor{Radovan Haluška}
\def\YearSubmitted{2022}

\def\Department{Department of Software Engineering}
\def\DeptType{Department}

\def\Supervisor{prof. RNDr. Tomáš Skopal, Ph.D.}
\def\SupervisorsDepartment{Department of Software Engineering}

\def\Advisor{Ing. Jan Brabec}

\def\StudyProgramme{Computer Science}
\def\StudyBranch{Artificial Intelligence}

\def\Dedication{
    I would like to thank the people that have been here for me through this journey.
}

\def\AbstractEN{
    Learning from imbalanced data has been a research topic studied for many years. There are two
    main approaches used today - data-level and algorithm-level methods. We set out to study
    resampling methods which belong to the category of data-level methods. These methods modify the
    training part of a dataset as opposed to algorithm-level methods, which modify a classifier
    itself. Resampling methods are further divided into oversampling and undersampling methods. It
    is challenging to know which group of methods performs better and which algorithms stand out
    the most. We conducted an experiment of unseen scale. We systematically and robustly compared
    sixteen preprocessing methods over eighteen imbalanced datasets and summarised the results in
    this thesis. The results show that oversampling methods outperformed most undersampling methods
    in both performance and preprocessing time.
}

\def\AbstractCS{
    Učenie sa z dát s nevyváženým pomerom tried je témou výskumu, ktorá sa skúma už mnoho rokov. V
    súčasnosti sa používajú dva hlavné prístupy – metódy na úrovni dát a metódy na úrovni
    algoritmov.  Rozhodli sme sa študovať metódy vzorkovania, ktoré patria do kategórie metód na
    úrovni dát. Tieto metódy modifikujú trénovaciu časť dát, na rozdiel od metód na úrovni
    algoritmov, ktoré modifikujú samotný klasifikátor. Metódy vzorkovania sa ďalej delia na metódy
    prevzorkovani a podvzorkovania. Je náročné vedieť, ktorá skupina metód funguje lepšie a ktoré
    algoritmy vynikajú najviac. Uskutočnili sme preto experiment nevídaného rozsahu. Systematicky a
    robustne sme porovnali šestnásť metód prevzorkovania nad osemnástimi datasetmi s nevyváženým
    pomerom tried a zhrnuli sme výsledky v tejto práci. Výsledky ukazujú, že metódy prevzorkovania
    prekonali väčšinu metód podvzorkovania z hľadiska výkonu aj času predspracovania.
}

\def\Keywords{
    {artificial intelligence}
    {machine learning}
    {imbalanced classification}
}

\def\InfoPageFont{}
